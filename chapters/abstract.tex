\newpage
\thispagestyle{plain}
\begin{center}
    \textbf{Περίληψη}
\end{center}

Παρουσιάζεται ένας νέος τρόπος έκφρασης και επίλυσης προβλημάτων μεταφοράς (advection) με βάση την συνάρτηση ροής τους, με την γλώσσα προγραμματισμού Julia, αποσκοπώντας στην άμεση διατύπωσή τους όσο γίνεται πιο κοντά στον μαθηματικό συμβολισμό τους.
Δίνεται έντονη έμφαση στην επίλυση των εξισώσεων συμπιεστής ροής του Euler.
Επιδυκνείεται πως τόσο η γλώσσα προγραμματισμού Julia, αλλά και το οικοσύστημα υπολογιστικών πακέτων που έχουν αναπτυχθεί γύρω από τις σχεδιαστικές αρχές της είναι ιδιαίτερα συμβατά με την ανάπτυξη αριθμητικών αλγορίθμων.
Εν συνεχεία, παρουσιάζεται ο σχεδιασμός και η ανάπτυξη ενός νέου πακέτου Julia που επιλύει τέτοια προβλήματα, ενώ παρέχει επιτάχυνση με χρήση καρτών γραφικών που υποστηρίζουν το μοντέλο προγραμματισμού CUDA.
Δίνονται παραδείγματα προβλημάτων, ενώ επαληθεύεται η αριθμητική ορθότητα του επιλυτή και ερευνάται η απόδοσή του.
Τέλος, ταυτοποιούνται οι περιορισμοί του πακέτου αυτού, ενώ προτείνονται μελλοντικές κατευθύνσεις ανάπτυξής του.

\begin{center}
    \textbf{Abstract}
\end{center}

A new way of expressing and solving advection problems is presented, based on their associated flux function, using the Julia programming language, with the goal of directly expressing them in a format closely resembling their mathematical description.
Particular emphasis is given to the solution of the compressible Euler flow equations.
The suitability of the Julia programming language as well as its numerical package ecosystem for the development of numerical algorithms is demonstrated.
The design and development of a new Julia package is shown, which can solve such problems and provide GPU acceleration using CUDA-compatible GPUs.
Example problems are given, and the numerical correctness and performance of the solver are examined.
Finally, the package's limitations are identified, and future development directions are suggested.

\newpage
