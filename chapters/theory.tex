\chapter{Θεωρητικό υπόβαθρο}
\label{chapter:theory}

\section{Υπολογιστική ρευστομηχανική}

Όπως αναφέρθηκε και στην εισαγωγή, η υπολογιστική ρευστομηχανική αναφέρεται στην επίλυση προβλημάτων ρευστομηχανικής με αριθμητικές μεθόδους.
Σε αυτήν την ενότητα θα ασχοληθούμε με την φυσική μοντελοποίηση των προβλημάτων που θέλουμε να λύσουμε, βήμα απαραίτητο πριν προχωρήσουμε στην αριθμητική μέθοδο.

Αρχικά, αξίζει να αναφέρουμε ότι στην παρούσα εργασία ασχολούμαστε με την περιγραφή των ρευστών σε μακροσκοπική κλίμακα.
Αυτό σημαίνει ότι θεωρούμε ότι μπορούμε να περιγράψουμε το ρευστό ως ένα \emph{συνεχές μέσο} (continuum) με σημειακές ιδιότητες (πυκνότητα, ταχύτητα, πίεση κλπ), χωρίς να ασχολούμαστε με την μοριακή φύση του ρευστού.

Απαραίτητη προϋπόθεση για αυτήν την παραδοχή είναι η κλίμακα του προβλήματος να είναι σημαντικά μεγαλύτερη από την μέση ελεύθερη διαδρομή (mean free path).
Μία αποδεκτή ποσοτικοποίηση αυτής της συνθήκης είναι να ισχύει $\mathrm{Kn} < 0.01$, όπου $\mathrm{Kn}$ ο αριθμός \emph{Knudsen} που ορίζεται ως:

\begin{equation*}
    \mathrm{Kn} = \frac{\lambda}{L}
\end{equation*}

όπου $\lambda$ η μέση ελεύθερη διαδρομή (mean free path) και L ένα χαρακτηριστικό μήκος του προβλήματος υπό μελέτη.

Με την παραδοχή αυτή του συνεχούς μέσου, η μοντελοποίηση της κίνησης των ρευστών κατά βάσει γίνεται με την εξίσωση ορμής Navier-Stokes, οι οποία σε σημειακή μορφή (point form) έχει την εξής μορφή:

\begin{equation}
    \rho \frac{\mathrm{D} \mathbf{u}}{\mathrm{D}t} = \nabla \cdot \mathbf{\sigma} + \mathbf{f}
    \label{eq:cauchy}
\end{equation}

όπου:

\begin{itemize}
    \item $\rho$ είναι η πυκνότητα,
    \item $\mathbf{u}$ είναι το διάνυσμα της ταχύτητας,
    \item $\frac{\mathrm{D}}{\mathrm{D}t} = \frac{\partial}{\partial t} + (\mathbf{u} \cdot \nabla )$ είναι η υλική παράγωγος (material derivative),
    \item $\mathbf{\sigma}$ είναι ο τανυστής της τάσης (stress tensor), και
    \item $\mathbf{f}$ είναι η τοπική δύναμη που τυχόν ασκείται σε κάθε σημείο του ρευστού, π.χ. για ένα ρευστό σε σταθερό, ομοιόμορφο βαρυτικό πεδίο $\mathbf{f} = \rho \mathbf{g}$ όπου $\mathbf{g}$ το διάνυσμα της βαρυτικής επιτάχυνσης.
\end{itemize}

Αυτή η εξίσωση περιγράφει την διατήρηση/μεταβολή της ορμής του ρευστού.
Για τον πλήρη ορισμό του προβλήματος και την επίλυσή του, χρειάζεται επίσης να ορίσουμε δύο ακόμη εξισώσεις.
Αυτές κατά κανόνα είναι:

\begin{itemize}
    \item Μία εξίσωση που προκύπτει από την αρχή διατήρησης της μάζας, εφόσον μπορούμε να κάνουμε μία τέτοια υπόθεση:
        \begin{equation*}
            \frac{\partial \rho}{\partial t} + \nabla \cdot \left( \rho \mathbf{u} \right) = 0
        \end{equation*}
    \item Μία εξίσωση που προκύπτει από την διατήρηση της ενέργειας.
\end{itemize}

Αυτές οι τρεις εξισώσεις μαζί συνήθως αναφέρονται ως οι \emph{εξισώσεις Navier-Stokes}.

Αν και οι εξισώσεις Navier-Stokes περιγράφουν την συμπεριφορά των ρευστών σε ένα πολύ μεγάλο εύρος προβλημάτων, είναι συνηθισμένο να χωρίζουμε τα προβλήματα σε υποκατηγορίες προκειμένου να επιλέξουμε έναν κατάλληλο αλγόριθμο επίλυσης.
Ένα κριτήριο διαχωρισμού είναι η τάξη μεγέθους του αριθμού Reynolds της ροής, που ορίζεται ως:

\begin{equation*}
    \mathrm{Re} = \frac{\rho U_{\infty} L}{\mu} = \frac{U_\infty L}{\nu}
\end{equation*}

όπου:

\begin{itemize}
    \item $\rho$ είναι η πυκνότητα του ρευστού,
    \item $U_{\infty}$ είναι η ταχύτητα ελεύθερης ροής,
    \item $L$ είναι το χαρακτηριστικό μήκος του προβλήματος,
    \item $\mu$ είναι το ιξώδες, και
    \item $\nu$ είναι το κινηματικό ιξώδες (kinematic viscosity) $\nu = \mu / \rho$.
\end{itemize}

Στην παρούσα εργασία θα ασχοληθούμε με μία υποκατηγορία προβλημάτων για την οποία $\mathrm{Re} \to \infty$.
Πρακτικά αυτό σημαίνει ότι η αδράνεια του ρευστού έχει πολύ μεγαλύτερη επιρροή στην συμπεριφορά του από ότι το ιξώδες, και συνεπώς μπορούμε να αγνοήσουμε την επιρροή του δεύτερου στο μοντέλο μας, με άλλα λόγια να θεωρήσουμε ότι η ροή είναι ατριβής (inviscid).
Αν και αυτή η παραδοχή μπορεί να φαίνονται αυθαίρετη, είναι μία καλή προσέγγιση στην περίπτωση διηχητικών και υπερηχητικών αεροδυναμικών ροών.
Η προσέγγιση παύει να είναι ικανοποιητική καθώς η ροή γίνεται υπερυπερηχητική (hypersonic), δηλαδή για αριθμό Mach μεγαλύτερο από 5, όπου φαινόμενα όπως ο ιονισμός του αερίου αρχίζουν να γίνονται σημαντικά.

Μία άλλη κατηγοριοποίηση που μπορούμε να κάνουμε, είναι αυτή της \emph{συμπιεστότητας} (compressibility) του ρευστού.
Για την περίπτωση των υγρών αλλά και ροών αερίων με αριθμό Mach μικρότερο από 0.3, μπορούμε πολλές φορές να θεωρήσουμε ότι η πυκνότητα είναι σταθερή.
Με αυτήν την παραδοχή, η εξίσωση διατήρησης της μάζας γίνεται

\begin{equation*}
    \nabla \cdot \mathbf{u} = 0
\end{equation*}

δηλαδή το πεδίο της ταχύτητας είναι ασυμπίεστο.
Σε αυτήν την περίπτωση οι μέθοδοι επίλυσης είναι διαφορετικοί, καθώς χρειάζεται να βρεθεί ένας τρόπος να υπολογιστεί η αλλαγή του πεδίου πίεσης, η οποία διαφορετικά προκύπτει από την εξίσωση κατάστασης του αερίου.
Μία γνωστή μέθοδος είναι η μέθοδος SIMPLE.

Όπως αναφέραμε προηγουμένως όμως, στην παρούσα εργασία θα ασχοληθούμε με την επίλυση προβλημάτων ροής για τα οποία ο αριθμός Mach είναι μεγαλύτερος από 0.3, καθώς μιλάμε για διηχητικές και υπερηχητικές ροές, και συνεπώς θα θεωρήσουμε ότι το αέριο είναι συμπιεστό.

\section{Συμπιεστή ροή χωρίς ιξώδες}

Πριν προχωρήσουμε στην μαθηματική περιγραφή των συμπιεστών ροών χωρίς ιξώδες, αξίζει να αναφέρουμε ένα φυσικό χαρακτηριστικό των συμπιεστών ροών με αριθμό Mach κοντά ή πάνω από 1, το οποίο θα πρέπει να λάβουμε υπόψιν μας στην σχεδίαση του αλγόριθμου επίλυσης.

Το φαινόμενο αυτό είναι το λεγόμενο \emph{κρουστικό κύμα} (shockwave), το οποίο αναφέρεται σε μία απότομη αλλαγή στις τιμές των ροϊκών πεδίων ή οποία διαδίδεται ως κύμα με την τοπική ταχύτητα του ήχου.
Ένα τυπικό παράδειγμα είναι τα κρουστικά κύματα που σχηματίζονται πάνω στις ακμές προσβολής (leading edge) των πτερυγίων υπερηχητικών αεροσκαφών, η τα σφαιρικά κύματα που σχηματίζονται με την ανατίναξη (detonation) μίας ποσότητας εκρηκτικών υλών.
Τα κρουστικά κύματα έχουν πολύ μικρό πάχος (συγκρίσιμο με την μέση ελεύθερη διαδρομή), και συνεπώς κατά την μαθηματική περιγραφή της λύσης μας χρειάζεται να τα αντιμετωπίσουμε ως ασυνέχειες.

Η μαθηματική περιγραφή συμπιεστών ροών χωρίς ιξώδες προκύπτει από τις εξισώσεις Navier-Stokes εάν αγνοήσουμε τους όρους του ιξώδους.
Με άλλα λόγια, θεωρούμε ότι η μοναδική δύναμη που ασκείται στο ρευστό είναι αυτή της πίεσης, δηλαδή ο τανυστής τάσης $\sigma$ γίνεται:

\begin{equation*}
    \mathbf{\sigma} = -p\mathbf{I} \Rightarrow \nabla \cdot \mathbf{\sigma} = -\nabla p
\end{equation*}

Θα θεωρήσουμε επίσης ότι η ροή είναι αδιαβατική, η επίδραση της βαρύτητας είναι αμελητέα και ότι δεν υπάρχει κάποιο άλλο πεδίο δύναμης που να ασκεί δύναμη στο ρευστό, δηλαδή $\mathbf{f} = 0$.
Συνδιάζοντας έτσι την \eqref{eq:cauchy} με την εξίσωση διατήρησης της μάζας και μία εξίσωση διατήρησης ενέργειας που προκύπτει από τον πρώτο νόμο της θερμοδυναμικής, προκύπτουν οι \emph{εξισώσεις του Euler}, σε σημειακή μορφή:

\begin{align}\label{eq:euler}
    \frac{\mathrm{D}\rho}{\mathrm{D}t} &= -\rho \nabla \cdot \mathbf{u} \\
    \frac{\mathrm{D} \mathbf{u}}{\mathrm{D}t} &= -\frac{\nabla p}{\rho} \\
    \frac{\mathrm{D} e_i}{\mathrm{D}t} &= -\frac{p}{\rho} \nabla \cdot \mathbf{u}
\end{align}

όπου $e_i$ η εσωτερική ενέργεια του αερίου ανα μονάδα μάζας, $e_i = c_v T$ όπου $T$ η απόλυτη θερμοκρασία.

Χρειαζόμαστε όμως μία ακόμη εξίσωση για τον υπολογισμό της πίεσης.
Ορίζοντας την συνολική ενέργεια ανα μονάδα μάζας $e = e_i + \frac{1}{2} |\mathbf{u}|^2$, για ιδανικά αέρια έχουμε

\begin{align*}
    p &= \rho R T \\
      &= \rho R \frac{e - \frac{1}{2} \mathbf{u}|^2}{c_v} \\
      &= \rho R \frac{\gamma - 1}{R} \left( e - \frac{1}{2} |\mathbf{u}|^2 \right) \\
      &= \rho \left(\gamma - 1\right) \left( e - \frac{1}{2} |\mathbf{u}|^2 \right) \\
\end{align*}

όπου $R$ η \emph{ειδική θερμοδυναμική σταθερά} (specific gas constant) και $\gamma = c_p / c_v$ ο λόγος των ειδικών θερμοτήτων (specific heat ratio).

Μπορούμε επίσης να εκφράσουμε τις εξισώσεις ως ένα πρόβλημα μεταφοράς (advection), όπου μία ποσότητα $\mathbf{U}$ μεταφέρεται με το διάνυσμα ροής (flux) $\mathbf{F}$, δηλαδή

\begin{equation*}
    \frac{\partial \mathbf{U}}{\partial t} + \nabla \cdot \mathbf{F} = 0
\end{equation*}

όπου

\begin{equation*}
    \mathbf{U} = 
        \begin{bmatrix}
            \rho \\
            \rho \mathbf{u} \\
            \rho e
        \end{bmatrix}
    , \ 
    \mathbf{F} =
        \begin{bmatrix}
            \rho \mathbf{u} \\
            \mathbf{u} \otimes \left( \rho \mathbf{u} \right) + p \mathbf{I} \\
            (e + p) \mathbf{u}
        \end{bmatrix}
\end{equation*}

ή, για την περίπτωση των δύο διαστάσεων που θα μας απασχολήσει και στην συνέχεια

\begin{equation*}
    \frac{\partial \mathbf{U}}{\partial t} + \frac{\partial \mathbf{F}}{\partial x} + \frac{\partial \mathbf{G}}{\partial y} = 0
\end{equation*}

όπου

\begin{equation*}
    \mathbf{U} = 
        \begin{bmatrix}
            \rho \\
            \rho u \\
            \rho v \\
            \rho e
        \end{bmatrix}
    , \ 
    \mathbf{F} =
        \begin{bmatrix}
            \rho u \\
            \rho u^2 + p \\
            \rho u v \\
            (e + p) u
        \end{bmatrix}
    , \ 
    \mathbf{G} =
        \begin{bmatrix}
            \rho v \\
            \rho u v \\
            \rho v^2 + p \\
            (e + p) v 
        \end{bmatrix}
\end{equation*}
.

\section{Η μέθοδος πεπερασμένων όγκων 1ης τάξης}
\label{seq:fvm-mol}

Οι παραπάνω εξισώσεις περιγράφουν την συμπεριφορά ενός ρευστού σε κάθε σημείο.
Για να είναι το πρόβλημα καλά ορισμένο (well-posed) χρειάζεται επίσης να ορίσουμε τις οριακές συνθήκες που πρέπει να ικανοποιεί η λύση, και κατά συνέπεια να ορίσουμε πλήρως τον χώρο (domain) στον οποίο η λύση ορίζεται.

Έστω λοιπόν ότι η λύση ενός τέτοιο προβλήματος είναι η συνάρτηση $U$ η οποία ορίζεται σε κάθε σημείο ενός πεπερασμένου χώρου $\Omega \in \mathbb{R}^2$.
Με άλλα λόγια $U : \Omega \rightarrow \mathbb{R}^n$ όπου $n$ ο αριθμός των βαθμωτών πεδίων που ορίζουν την λύση.
Για τις εξισώσεις Euler σε δύο διαστάσεις όπως ορίζονται στην προηγούμενη ενότητα $n = 4$.

Για την αριθμητική προσέγγιση της λύσης $U$, το πρώτο βήμα είναι να βρούμε έναν τρόπο αναπαράστασης της λύσης με έναν πεπερασμένο αριθμό (έστω $N$) συντελεστών, έτσι ώστε αυτή να μπορεί να αποθηκευθεί και να επεξεργαστεί από έναν ηλεκτρονικό υπολογιστή.
Έστω $U_i \in \mathbb{R}^n$ ο συντελεστής $i$.
Μπορούμε να ορίσουμε την αριθμητική προσέγγιση της λύσης $\hat{U}$ ως τον γραμμικό συνδιασμό $N$ συναρτήσεων βάσης $\phi_i : \Omega \rightarrow \mathbb{R}$:

\begin{equation}
    \hat{U} = \sum_{i = 0}^{N} U_i \phi_i
\end{equation}

Εάν $V$ ο διανυσματικός χώρος (vector space) όλων των συναρτήσεων από το $\Omega$ στο $\mathbb{R}^n$, τότε, για πεπερασμένο $N$, οι συναρτήσεις βάσης ορίζουν έναν υποχώρο του $V$, και άρα η αριθμητική προσέγγιση $\hat{U}$ δεν μπορεί να αναπαραστήσει πλήρως την πραγματική λύση $U$ εκτός και αν αυτή βρίσκεται στον υποχώρο που ορίζουν οι συναρτήσεις βάσης.

Για να ορίσουμε της συναρτήσεις βάσης που θα χρησιμοποιήσουμε στην μέθοδο πεπερασμένων όγκων, ξεκινάμε χωρίζοντας τον χώρο $\Omega$ σε $N$ υποχώρους $S_i$ έτσι ώστε:

\begin{equation}
    \Omega = \bigcup_{i = 0}^{N} S_i
\end{equation}

Τους υποχώρους αυτούς θα τους αποκαλούμε \emph{πεπερασμένους όγκους} (finite volumes) ή \emph{κελιά} (cells).

Μπορούμε να ορίσουμε τώρα τις συναρτήσεις βάσης της μεθόδου πεπερασμένων όγκων πρώτης τάξης ως:

\begin{equation}
    \phi_i(\mathbf{x}) =
        \begin{cases}
            1 & \mathbf{x} \in S_i \\
            0 & \mathbf{x} \notin S_i
        \end{cases}
\end{equation}

Πρακτικά αυτό σημαίνει ότι υποθέτουμε ότι η λύση είναι σταθερή στο εσωτερικό κάθε πεπερασμένου όγκου, αλλά ασυνεχής ανάμεσά τους.
Καθώς η προσέγγιση $\hat{U}$ είναι γραμμικός συνδιασμός αυτών των μη συνεχών συναρτήσεων βάσης, η προσέγγιση θα είναι και αυτή ασυνεχής.

\begin{figure}
    \centering
    \tdplotsetmaincoords{70}{80}
    \begin{tikzpicture}[scale=2, tdplot_main_coords]
        \coordinate (A) at (-2, -1.3, 0);
        \coordinate (B) at (1, -1, 0);
        \coordinate (C) at (-1, 1, 0);
        \coordinate (D) at (1.5, 1.5, 0);
        \coordinate (Ah) at (-2, -1.3, 2.0);
        \coordinate (Bh) at (1, -1, 2.0);
        \coordinate (Ch) at (-1, 1, 2.0);
        \coordinate (Bl) at (1, -1, 1.2);
        \coordinate (Cl) at (-1, 1, 1.2);
        \coordinate (Dl) at (1.5, 1.5, 1.2);

        \coordinate (S1) at (-0.67, -0.43, 0);
        \coordinate (S1h) at (-0.67, -0.43, 2.0);

        \coordinate (S2) at (0.5, 0.5, 0);
        \coordinate (S2h) at (0.5, 0.5, 1.2);


        \draw[dashed] (A) -- (B) -- (C) -- cycle;
        \draw[dashed] (C) -- (D) -- (B);

        \draw (Ah) -- (Bh) -- (Ch) -- cycle;
        \draw (Bl) -- (Cl) -- (Dl) -- cycle;

        \draw[dotted] (Ah) -- (A) (Bl) -- (B) (Cl) -- (C) (Dl) -- (D);
        \draw (Bh) -- (Bl) (Ch) -- (Cl);

        \draw (S1) node{$S_1$} (S2) node{$S_2$};
        \draw (S1h) node{$U_1 \phi_1$} (S2h) node{$U_2 \phi_2$};

        \draw (Ch) +(0, 0, -0.4) node[right]{$\hat{U} = \sum_i^N U_i \phi_i$};
    \end{tikzpicture}
    \caption{Στην μέθοδο πεπερασμένων όγκων πρώτης τάξης υποθέτουμε ότι η λύση είναι σταθερή σε κάθε πεπερασμένο όγκο. Αυτό οδηγεί στον σχηματισμό ασυνεχειών στην επιφάνεια της λύσης $\hat{U}$ στα όρια ανάμεσα στους πεπερασμένους όγκους, αλλά επιτρέπει επίσης την καλή αναπαράσταση ασυνεχειών στην λύση όπως είναι π.χ. τα κρουστικά κύματα.}
    \label{fig:fvm-basis}
\end{figure}

Παραμένει όμως το πρόβλημα υπολογισμού των συντελεστών $U_i$.
Οι συντελεστές αυτοί είναι συναρτήσεις του χρόνου, καθώς μέχρι στιγμής έχουμε διακριτοποιήσει το πρόβλημα μόνο στον χώρο.
Η μέθοδος πεπερασμένων όγκων θα μας επιτρέψει να κατασκευάσουμε ένα σύστημα διαφορικών εξισώσεων που περιγράφει την χρονική εξέλιξη αυτών των συντελεστών, και μας δίνει σε κάθε χρονική στιγμή έναν τρόπο υπολογισμού των χρονικών παραγώγων $\frac{\mathrm{d}U_i}{\mathrm{d}t}$.
Με τις χρονικές παραγώγους διαθέσιμες, μπορούμε να λύσουμε το σύστημα διαφορικών εξισώσεων που προκύπτει με έναν από τους πολλούς αλγόριθμους χρονικής ολοκλήρωσης που υπάρχουν, επιλύοντας είτε μέχρι μία χρονική στιγμή $t_1$, για την περίπτωση ενός μεταβατικού προβλήματος (transient problem), είτε μέχρι οι χρονικές παράγωγοι να πλησιάσουν ικανοποιητικά κοντά στο μηδέν για ένα πρόβλημα μόνιμης κατάστασης (steady state problem).
Υπάρχουν πολλοί τέτοιοι αλγόριθμοι, με τον πιό απλό να είναι ο αλγόριθμος χρονικής ολοκλήρωσης του Euler όπου $u^{t + 1} = u^{t} + \frac{\mathrm{d} u}{\mathrm{d}t} \mathrm{d} t$.
Καθώς η επιλογή αυτή δεν αλλάζει την μέθοδο υπολογισμού των παραγώγων που θα περιγράψουμε παρακάτω, στο σημείο αυτό δεν θα επεκταθούμε περαιτέρω στο θέμα της χρονικής ολοκλήρωσης.

Η μέθοδος πεπερασμένων όγκων μας επιτρέπει να επιλύσουμε προβλήματα της μορφής

\begin{equation}
    \label{eq:advection-point-form}
    \frac{\partial \mathbf{U}}{\partial t} + \nabla \cdot \mathbf{F} = 0
\end{equation}

Η εξίσωση αυτή θα πρέπει να ισχύει σε κάθε σημείο του χώρου $\Omega$ και, κατά συνέπεια, σε κάθε σημείο ενός πεπερασμένου όγκου $S_i$ για κάθε $i$.
Ολοκληρώνοντας την εξίσωση \eqref{eq:advection-point-form} στο εσωτερικό ενός πεπερασμένου όγκου $S_i$ έχουμε:

\begin{equation*}
    \label{eq:fvm-int}
    \int_{S_i} \left( \frac{\partial U}{\partial t} + \nabla \cdot \mathbf{F} \right) \mathrm{d} S = 
    \int_{S_i} \frac{\partial U}{\partial t} \mathrm{d} S
    + \int_{S_i} \left( \nabla \cdot \mathbf{F} \right) \mathrm{d} S = 0
\end{equation*}

Ξεκινώντας με τον πρώτο όρο της εξίσωσης \eqref{eq:fvm-int}, και θυμίζοντας ότι $U = \sum_{i = 0}^N U_i \phi_i$, έχουμε:

\begin{align}
    &\int_{S_i} \frac{\partial U}{\partial t} \mathrm{d} S \notag \\
    = &\int_{S_i} \frac{\partial}{\partial t}\left( \sum_{j = 0}^{N} U_j \phi_j \right) \mathrm{d} S \notag \\
    = &\int_{S_i} \frac{\partial}{\partial t}\left( U_i \phi_i \right) \mathrm{d} S \quad \text{καθώς $\phi_j = 0$ αν $i \neq j$} \notag \\
    = &A \frac{\mathrm{d}U_i}{\mathrm{d}t} \quad \text{καθώς $\phi_i = 1$ στο εσωτερικό του $S_i$} \label{eq:fvm-lhs}
\end{align}

όπου $A = \int_{S_i} \mathrm{d} S$ η επιφάνεια (ή όγκος στις τρεις διαστάσεις) του πεπερασμένου όγκου $S_i$.

Συνεχίζοντας με τον δεύτερο όρο της εξίσωσης \eqref{eq:fvm-int}, χρησιμοποιώντας το θεώρημα της απόκλισης (divergence theorem) έχουμε:

\begin{equation}
    \label{eq:fvm-rhs}
    \int_{S_i} \left( \nabla \cdot \mathbf{F} \right) \mathrm{d} S 
    = \oint_{\partial S_i} \left( \mathbf{F} \cdot \mathbf{n} \right) \mathrm{d} l
\end{equation}

όπου $\mathbf{n}$ είναι ένα διάνυσμα κάθετο στο όριο $\partial S_i$ με μήκος 1, και $l$ το διαφορικό μήκους (ή επιφανείας στις τρεις διαστάσεις).

Συνδιάζοντας τις εξισώσεις \eqref{eq:fvm-lhs} και \eqref{eq:fvm-rhs} έχουμε την εξής σχέση για τον υπολογισμό της χρονικής παραγώγου του συντελεστή $U_i$:

\begin{equation}
    \label{eq:fvm-update-cont}
    \frac{\mathrm{d} U_i}{\mathrm{d} t} = -\frac{1}{A} \oint_{\partial S_i} \left( \mathbf{F} \cdot \mathbf{n} \right) \mathrm{d} l
\end{equation}

Εδώ όμως έχουμε ένα ακόμη πρόβλημα: Η συνάρτηση της ροής $\mathbf{F}$ είναι συνήθως συνάρτηση της λύσης $U$, αλλά η προσέγγιση της λύσης είναι ασυνεχής πάνω στο όριο ανάμεσα σε δύο πεπερασμένους όγκους.
Χρειάζεται να βρούμε έναν τρόπο λοιπόν, να αντικαταστήσουμε την συνάρτηση $\mathbf{F}(U)$ με μία άλλη $\mathbf{F}\left(U_1, U_2\right)$ που υπολογίζει την ροή λαμβάνοντας υπόψιν τις τιμές της λύσης στις 2 πλευρές του ορίου πάνω στο οποίο ολοκληρώνουμε.

Μία γνωστή μέθοδος, είναι η μέθοδος Godunov \cite[p. ~105]{Knight2006}. Αν εστιάσουμε πάνω σε ένα σημείο του ασυνεχούς ορίου, και αγνοήσουμε προς το παρόν τους υπόλοιπους πεπερασμένους όγκους, μπορούμε να θεωρήσουμε ότι έχουμε ένα μονοδιάστατο πρόβλημα αρχικών τιμών (initial value problem/IVP), του οποίου η λύση είναι η εξέλιξη στον χρόνο μιας τέτοιας ασυνέχειας.
Το πρόβλημα αυτό λέγεται \emph{πρόβλημα Riemann}, και υπάρχουν αρκετοί αλγόριθμοι επίλυσής του, είτε ακριβώς (π.χ. Godunov), είτε προσεγγιστικά (π.χ. μέθοδος Roe με γραμμικοποίηση, HLLC, HLLE, κ.α.).
Οι αλγόριθμοι αυτοί έχουν ως αποτέλεσμα τον υπολογισμό της ροής $\mathbf{F}\left(U_1, U_2\right)$ που ψάχνουμε, με έναν αναπόφευκτο συμβιβασμό ανάμεσα σε ταχύτητα και ακρίβεια.

Μία άλλη μέθοδος είναι αυτή του διαχωρισμού των διανυσμάτων εκροής (Flux Vector Splitting/FVS).
Εδώ χωρίζουμε την ροή $\mathbf{F}$, σε δύο διανύσματα ως εξής:

\begin{equation}
    \mathbf{F} = \mathbf{F}_{+} + \mathbf{F}_{-}
\end{equation}

όπου:
\begin{itemize}
    \item $\mathbf{F}_{+}$ είναι η ροή που προκύπτει από την διάδοση κυμάτων στην θετική κατεύθηνση (έξω από τον όγκο ελέγχου), και
    \item $\mathbf{F}_{-}$ είναι η ροή που προκύπτει από την διάδοση κυμάτων στην αρνητική κατεύθηνση (προς το εσωτερικό του όγκου ελέγχου)
\end{itemize}

Για τον χωρισμό των διανυσμάτων εκροής, χρειάζεται να βρούμε της χαρακτηριστικές (characteristics) της εξίσωσης, οι οποίες μας δίνουν την κατεύθηνση και ταχύτητα διάδοσης των κυμάτων.
Για τις εξισώσεις Euler σε μία διάσταση, προκύπτουν τρεις ταχύτητες διάδοσης $u$, $u + \alpha$, και $u - \alpha$, όπου $u$ η ταχύτητα της ροής στην θετική κατεύθηνση και $\alpha$ η τοπική ταχύτητα του ήχου.
Υπάρχουν πολλοί αλγόριθμοι που βασίζονται στην μέθοδο FVS \cite{Toro2012}.

Για τις υπολογιστικές μελέτες της παρούσας εργασίας θα χρησιμοποιήσουμε την μέθοδο χωρισμού διανυσμάτων εκροής AUFS \cite{Sun2003}, αλλά η υπολογιστική μέθοδος της εργασίας επιτρέπει την χρήση άλλων μεθόδων υπολογισμού του $\mathbf{F}$ χωρίς ιδιαίτερο κόπο.

Βλέπουμε λοιπόν ότι η μέθοδος πεπερασμένων όγκων μας παρέχει τόσο μία επιλογή συναρτήσεων βάσης αλλά και μία μέθοδο υπολογισμού των συντελεστών $U_i$ έτσι ώστε η προσέγγιση $\hat{U}$ να είναι μία καλή προσέγγιση της πραγματικής λύσης $U$.
Η μετατροπή αυτή ενός συστήματος μερικών διαφορικών εξισώσεων (PDE) σε έναν σύστημα απλών διαφορικών εξισώσεων (ODE) και η επίλυση αυτού με συμβατικές μεθόδους είναι επίσης γνωστή ως η \emph{μέθοδος των γραμμών} (method of lines) \cite[p. ~352]{Solomon2015}.

Αν και δεν υπάρχει κάποιος θεωρητικός περιορισμός όσον αφορά το σχήμα των πεπερασμένων όγκων, στην πράξη χρησιμοποιούνται κατά βάση πολύγωνα.
Για προβλήματα δύο διαστάσεων, συνήθης επιλογές είναι τα τρίγωνα (λόγω απλότητας και ύπαρξης μεγάλης πληθώρας αλγορίθμων τριγωνοποίησης, π.χ. Delauney), τα τετράγωνα (κυρίως σε δομημένα πλέγματα (structured grids)), και τα εξάγωνα (λόγω δυνατότητας κάλυψης του χώρου με μικρότερο αριθμό κελιών).

\begin{figure}
    \centering
    \begin{tikzpicture}[scale=1.5]
        \coordinate (A) at (-2, -1.3);
        \coordinate (B) at (1, -1);
        \coordinate (C) at (-1, 1);
        \coordinate (D) at (1.5, 1.5);
        \coordinate (CB) at (0, 0);
        \coordinate (AB) at (-0.5, -1.15);
        \coordinate (AC) at (-1.5, -0.15);

        \draw (A) -- (B) -- (C) -- cycle;
        \draw (C) -- (D) -- (B);

        \draw[dashed] (C) -- +(80:1) (C) -- +(160:1)
            (A) -- +(110:1) (A) -- +(170:1) (A) -- +(230:1) (A) -- +(300:1)
            (B) -- +(240:1) (B) -- +(330:1) (B) -- +(20:1)
            (D) -- +(320:1) (D) -- +(0:1) (D) -- +(60:1) (D) -- +(130:1);

        \fill (CB) circle[radius=0.05] (AB) circle[radius=0.05] (AC) circle[radius=0.05];

        \draw[-{Stealth[scale=2]}] (AB) -- +(0.06, -0.6) node[below]{$\mathbf{\hat{n}}_1$};
        \draw[-{Stealth[scale=2]}] (CB) -- +(0.5, 0.5) node[below right]{$\mathbf{\hat{n}}_2$};
        \draw[-{Stealth[scale=2]}] (AC) -- +(-0.575, 0.25) node[above]{$\mathbf{\hat{n}}_3$};

        \draw (AB) +(-0.25, 0.5) node[above]{$S_i$};

        \draw (CB) +(-0.5, 0.5) node[below left]{$U_i$};
        \draw (CB) +(-0.5, 0.5) node[above right]{$U_{+,2}$};

        \draw (CB) +(0.05, 0) node[right]{$L_2$};
        \draw (AB) node[above]{$L_1$};
        \draw (AC) node[below right]{$L_3$};
        \draw[-{Stealth[scale=2]}] (CB) -- +(0.4, 0.8) node[above right]{$\mathbf{F}$};
    \end{tikzpicture}
    \caption{Σχηματική αναπαράσταση του υπολογισμού της ροής ανάμεσα σε δύο γειτονικά τριγωνικά στοιχεία στον διδιάστατο χώρο.}
    \label{fig:fvm-triangle-flux}
\end{figure}

Η χρήση πολυγώνων κάνει τον υπολογισμό του ολοκληρώματος της εξίσωσης \eqref{eq:fvm-update-cont} ιδιαίτερα εύκολο. Τόσο οι τιμές της λύσης που χρησιμοποιούνται στον υπολογισμό του $\mathbf{F}$, τόσο και το μοναδιαίο διάνυσμα $\mathbf{\hat{n}}$ στο οποίο προβάλλουμε το διάνυσμα ροής $\mathbf{F}$ είναι σταθερά κατά μήκος τις κάθε πλευράς του πολυγώνου. Ο τρόπος υπολογισμού της ροής σε κάθε πλευρά ενός τριγώνου στο διδιάστατο χώρο φαίνεται στο σχήμα \ref{fig:fvm-triangle-flux}. Έτσι το ολοκλήρωμα μετατρέπεται σε απλό άθροισμα, με έναν όρο για κάθε πλευρά του πολυγώνου:

\begin{equation}
    \frac{\mathrm{d} U_i}{\mathrm{d} t} = -\frac{1}{A_i}
    \sum_{j = 0}^{K} \mathbf{F} \left( U_i, U_{+,j} \right) \cdot \mathbf{\hat{n}}_{j} L_{j}
\end{equation}

όπου:

\begin{itemize}
    \item $K$ ο αριθμός των πλευρών του πολυγώνου του πεπερασμένου όγκου $S_i$,
    \item $A_i$ η επιφάνεια (ή ο όγκος στις τρεις διαστάσεις) του πεπερασμένου όγκου $S_i$,
    \item $U_{+,j}$ ο συντελεστής της λύσης του $j$ γείτονα του όγκου $S_i$,
    \item $\mathbf{\hat{n}}_{j}$ το μοναδιαίο διάνυσμα κάθετο στο όριο $\partial S_i$ του όγκου $S_i$ με τον $j$ γείτονά του, και
    \item $L_j$ το μήκος (ή η επιφάνεια στις τρεις διαστάσεις) του ορίου αυτού.
\end{itemize}

